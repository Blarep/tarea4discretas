\documentclass[spanish, fleqn]{article}
\usepackage{babel}
\usepackage[utf8]{inputenc}
\usepackage{amsmath,amsfonts}
\usepackage{enumitem}
\usepackage[colorlinks, urlcolor=blue]{hyperref}
\usepackage{fourier}
\usepackage{tikz}
\usetikzlibrary{shapes.geometric}
\usetikzlibrary{positioning}
\usepackage{verbatim}
\usepackage[top = 2.5cm, bottom = 2cm, left = 2.5cm, right = 2.5cm]{geometry}
\usepackage{parskip}
\setlength{\parskip}{2mm}
\pagestyle{empty}
\newcommand{\num}{4}

\title{Estructuras Discretas \\
	Tarea \#\num \\
	``Cómo aprender a generar en una tarde''}
\author{Andrés Navarro \\ (201673001-K)}
\date{}
\begin{document}
	\maketitle
	\thispagestyle{empty}
	
    Utilice funciones generatrices ordinarias para responder las siguientes preguntas.
    
	\section{Recurrencia}
	Halle una fórmula explícita para \(a_n\) dada la siguiente recurrencia:
	
    $a_n = \binom{n}{2}+ 3a_{n-1} a_{0}=1$
    
    Se define la generatriz a buscar:
    \begin{equation*}
    A(z) = \sum_{n \geqslant 0} a_n z^n
    \end{equation*}
	
	Modificamos la expresión inicial:
	\begin{equation*}
	a_{n+1} = \binom{n+1}{2} + 3a_{n} \qquad a_{0}=1
	\end{equation*}
	
	Aplicando ogf en cada término:
	\begin{align*}
	\sum_{n \geqslant 0} a_{n+1}z^n &= \sum_{n \geqslant 0} \binom{n+1}{2}z^n +\sum_{n \geqslant 0} 3a_{n}z^n \\
	\frac{A(z)- a_{0}}{z} &= \sum_{n \geqslant 0} \binom{n+1}{2}z^n + 3A(z)
	\end{align*}
	Trabajando la sumatoria		
	\begin{align*}
	\sum_{n \geqslant 0} \binom{n+1}{2}z^n &= z\sum_{n \geqslant 0} \binom{n+1}{2}z^{n-1}\\
	\end{align*}
	Cambio de variable $k = n-1$
	\begin{align*}	
	\sum_{n \geqslant 0} \binom{n+1}{2}z^n&=z\sum_{k \geqslant 0} \binom{k+2}{2}z^k\\  
	&=z \cdot \frac{1}{(1-z)^3}
	\end{align*}
	Reemplazando en la expresión original
	\begin{align*}
	\frac{A(z)- 1}{z} &=z \cdot \frac{1}{(1-z)^3} + 3A(z)
	\end{align*}
	Ahora despejaremos $A(z)$
	\begin{align*}
	A(z) &=\frac{z^2}{(1-z)^3} + 3\cdot z\cdot A(z)+ 1\\
	A(z) (1-3z)&=\frac{z^2}{(1-z)^3}  +1\\
	A(z)&=\frac{z^2}{(1-z)^3(1-3z)} +  \frac{1}{(1-3z)}
	\end{align*}
	Utilizando fracciones parciales
	\begin{align*}
	A(z)&= -\frac{11}{8(3z-1)}+ \frac{1}{8(z-1)}+\frac{1}{4(z-1)^2}+\frac{1}{2(z-1)^3}\\
	&= \frac{11}{8(1-3z)}- \frac{1}{8(1-z)}+\frac{1}{4(1-z)^2}-\frac{1}{2(1-z)^3}\\
	&= \frac{11}{8}\sum_{n \geqslant 0} 3^n z^n -\frac{1}{8}\sum_{n \geqslant 0} z^n + \frac{1}{4} \sum_{n \geqslant 0} (n+1)z^n-\frac{1}{2} \sum_{n \geqslant 0} \frac{(n+1)(n+2)}{2}z^n
	\end{align*}
	Extraemos $a_n$ en cada expresión y obtenemos finalmente:
	\begin{align*}
	a_n &= \frac{11}{8} \cdot 3^n-\frac{1}{8}+\frac{1}{4} \cdot (n+1)-\frac{1}{4} \cdot (n+1)(n+2)\\
	a_n &= \frac{11 \cdot 3^n}{8} - \frac{1}{8}-\frac{(n+1)^2}{4}
	\end{align*}
	%% Pregunta 2
	\section{Safari}
	Un safari nuevo en la Selva Amazónica quiere armar un área mixta solo con animales herbívoros. El safari cuenta con los siguientes: elefantes, jirafas, ciervos y gansos. El área debe obedecer:
    \begin{itemize}
    \item Puede haber una cantidad libre de gansos.
    \item No puede haber más de dos elefantes.
    \item Los ciervos deben ir en grupos de a cuatro.
    \item No debe haber más de cinco jirafas.
    \end{itemize}
    Generatrices:
    
    Para los ganzos queda definida por:
    \begin{align*}
    G(z)=z^0+z^1+z^2+\dots+z^n=\sum_(n \geqslant 0) z^n =\frac{1}{1-z}
    \end{align*}
    Para los elefantes, dado que solo pueden haber 2 como máximo, se tiene que su generatriz solo puede llegar al término $z^2$, que es la siguiente:
    \begin{align*}
    E(z)=z^0+z^1+z^2
    \end{align*}
    Para los ciervos solo se necesitan los términos de la generatriz que sean múltiplos de 4:
    \begin{align*}
    C(z)=z^0+z^4+z^8+\dots+z^{4n}=\sum_{n \geqslant 0} (z^4)^n=\frac{1}{1-z^4}
    \end{align*}
    Finalmente para las girafas, su generatriz solo puede llegar al término $z^5$:
    \begin{align*}
    G(z)=z^0+z^1+z^2+z^3+z^4+z^5
	\end{align*}    
	Luego de esto, al multiplicar las cuatro generatrices obtendremos un polinomio que determinara las formas de ordenarlos según la cantidad de animales, esto se determina con el coeficiente que acompaña a un z(formas de ordenarlo) y el exponente al que esta elevado (cantidad de animales).
	
	Generatriz safari:
	\begin{align*}
	S(z)=\frac{1}{1-z} \cdot (z^0+z^1+z^2) \cdot \frac{1}{1-z^4} \cdot (z^0+z^1+z^2+z^3+z^4+z^5) 
	\end{align*}	
	Utilizando fracciones parciales:
	\begin{align*}
	S(z)= z^2-\frac{1}{2(z^2+1)}+3z+\frac{9}{z-1}+\frac{9}{2(z-1)^2}+6
	\end{align*}
	Manipulando las expresiones:
	\begin{align*}
	&z^2=\sum_{n=2} z^n\\
	&-\frac{1}{2(z^2+1)}=\frac{-1}{2}\sum_{n \geqslant 0} \binom{-1}{n}z^{2n}=\frac{-1}{2}\sum_{n\geqslant 0} (-1)^n\binom{n}{0}z^{2n}=\frac{-1}{2}\sum_{n\geqslant 0}(-1)^nz^{2n}\\
	&3z=3\sum_{n=1} z^n\\
	&\frac{9}{z-1+}=\frac{-9}{1-z}=-9\sum_{n \geqslant 0} z^n\\
	&\frac{9}{2(z-1)^2}=\frac{9}{2}\sum_{n \geqslant 0} (n+1)z^n\\
	&6=6z^0=6\sum_{n=0}z^n
	\end{align*}
	Reemplazando en $S(z)$:
	\begin{align*}
	S(z)=\sum_{n=2} z^n+\frac{-1}{2}\sum_{n {}\geqslant 0} (-1)^nz^{2n}+3\sum_{n=1} z^n-9\sum_{n \geqslant 0} z^n+\frac{9}{2}\sum_{n \geqslant 0} (n+1)z^n+6\sum_{n=0}z^n
	\end{align*}
    \begin{enumerate}
    \item[a)] ¿De cuántas formas se puede armar esta área si se quiere que hayan 20 animales en ella?
    
    En este caso nos intereza obtener todos los términos que esten elevados a 20, ya que queremos formar grupos de 20 animales. Entonces, en cada expresión de $S(z)$ intentamos obtener un $z^20$ para poder sumarlos todos.
    \begin{align*}
    S(z)&=\frac{-1}{2}(-1)^{10} \cdot z^{2(10)}- 9z^{20}+\frac{9}{2}(20+1)z^{20}\\
    &=z^{20}(\frac{-1}{2}-9+\frac{189}{2})\\
    &=z^{20}(85)
    \end{align*}
    Luego de esto concluimos que existen 85 formas distinas de armar la zona del safari.
    
    \item[b)] Se agrega una quinta regulación: si se decide incluir alguna especie en el área, se debe incluir más de un animal de dicha especie. ¿De cuántas formas se puede armar ahora el área con 20 animales?
    La generatriz de la nueva especie esta definida por:
    \begin{align*}
    N(z)=z^2+z^3+ \dots +z^n \\
    = -(z+1) + \sum_{n \geqslant 0} z^n\\
    =-(z+1)+\frac{1}{1-z}\\
    =\frac{z^2}{1-z}
    \end{align*}
    Multiplicamos por S(z) para obtener la nueva generatriz para ordenar a los animales:
    \begin{align*}
    F(z)&=N(z) \cdot S(z)\\
    	&=\frac{z^2}{1-z} \cdot \frac{1}{1-z} \cdot (z^0+z^1+z^2) \cdot \frac{1}{1-z^4} \cdot (z^0+z^1+z^2+z^3+z^4+z^5) 
    \end{align*}
    Ocupando calculadora de series de Taylor obtenemos:
    \begin{align*}
    F(z)=z^2+4z^3+10z^4+19z^5+32z^6+ \dots + 617z^{19} + 694z^{20}+775z^21+ \dots
	\end{align*}
	Con lo que determinamos que el número de formas de armar el safari con la quinta regulación es 694, ya que este número es el que acompaña a $z^20$.
    \end{enumerate}
	
	%% Pregunta 3
	\section{Da2 mágicos}
    Se tienen dos dados canónicos de 4 caras (\{1,2,3,4\}), un dado canónico de 6 caras (\{1,2,3,4,5,6\}), dos dados canónicos de 8 caras (\{1,2,3,4,5,6,7,8\}) y un dado desconocido. Determine, en cada caso, el dado desconocido (número de caras y los valores de cada cara):
    
    Generatrices:
    \begin{align*}
    D_{4}(z)&=z+z^2+z^3+z^4\\
    D_{6}(z)&=z+z^2+z^3+z^4+z^5+z^6 \\
    D_{8}(z)&=z+z^2+z^3+z^4+z^5+z^6+z^7+z^8
    \end{align*}
    
    \begin{itemize}
    \item[a)] Si se tiran los 2 dados canónicos de 4 caras y el dado desconocido, existe la misma posibilidad de que salgan cada uno de los valores que al tirar los 2 dados canónicos de 8 caras.
    Planteamos la igualdad como una multiplicación de generatrices:
    \begin{align*}
    D_d(z) \cdot D_4(z) \cdot D_4(z) &= D_8(z) \cdot D_8(z)\\
    D_d(z) \cdot (D_4(z))^2 &=(D_8(z))^2\\
    D_d(z)&=\frac{(D_8(z))^2}{(D_4(z))^2}\\
    D_d(z)&=\frac{(z+z^2+z^3+z^4+z^5+z^6+z^7+z^8)^2}{(z+z^2+z^3+z^4)^2}\\
    D_d(z)&=(z^4+1)^2\\
    D_d(z)&=z^8+2z^4+1
    \end{align*}
    Luego de despejar la generatriz del dado desconocido podemos determinar que este posee 4 caras, una con un valor de 8, dos con un valor de 4 y una con un valor igual a cero(\{8,4,4,0\}).
    
    \item[b)] Determine si existe un dado tal que si se tira el dado desconocido y el dado (\{0,1,2\}), exista el doble de posibilidad de que salga un 1, 3 y 5 que al tirar el dado canónico de 6 caras, y la misma posibilidad para los demás valores.
    
    El dado nuevo de 3 caras queda definido por la generatriz:
    \begin{align*}
    D_3(z)=1+z+z^2
    \end{align*}
    
    Ahora bien, se nos pide determinar un dado tal que multiplicado por el anterior, tengamos el doble de posibilidades de obtener 1,3 y 5 que al tirar un dado canónico de 6 caras. Esto se puede expresar modificandola genetratriz del dado canónico de 6 caras de la siguiente forma:
    \begin{align*}
    D_{6nuevo}(z)&=2z+z^2+2z^3+z^4+2z^5+z^6
    \end{align*}
    De esta forma conseguimos que las posibilidades de obtener 1,3 y 5 sean el doble que las demás debido a los coeficientes que los acompañan.
    Ahora planteamos la igualdad como una multiplicación de generatrices:
    \begin{align*}
    D_d(z) \cdot D_3(z)&=D_{6nuevo}(z)\\
    D_d(z) &= \frac{D_{6nuevo}(z)}{D_3(z)}\\
    D_d(z)&= \frac{2z+z^2+2z^3+z^4+2z^5+z^6}{1+z+z^2}    \\
    D_d(Z)&=z^4+z^3-z^2+2z
    \end{align*}
    Al ver esta generatriz, nos percatamos de que posee un coeficiente con un valor negativo, lo que indicaría que posee una cara negativa con valor igual a dos, situación que no puede ocurrir, por lo que no existe dado que cumpla la condición.
    \item[c)] ¿Existe un dado tal que al tirarlo con el dado \{1,2,2,3,3,4,4,5\}, tiene la misma posibilidad de que salgan cada uno de los valores que al tirar los dos dados canónicos de 8 caras? Si es así, encuentre el dado.
    
    El dado nuevo de 8 caras queda definido por la generatriz:
    \begin{align*}
    D_{nuevo}(z)=z+2z^2+2z^3+2z^4+z^5
    \end{align*}
    Planteamos la igualdad como una multiplicación de generatrices:
    \begin{align*}
    D_d(z) \cdot D_{nuevo}(z)&=(D_8(z))^2\\
    D_d(z)&= \frac{(D_8(z))^2}{D_{nuevo}(z)}\\
    D_d(z)&=\frac{(z+z^2+z^3+z^4+z^5+z^6+z^7+z^8)^2}{z+2z^2+2z^3+2z^4+z^5}\\
    D_d(z)&=z^11+z^9+2z^7+2z^5+z^3+z
    \end{align*}
    De esta forma, al despejar la generatriz del dado desconocido podemos determinar que este posee 8 caras (\{11,9,7,7,5,5,3,1\}).
	\end{itemize}
\end{document}
