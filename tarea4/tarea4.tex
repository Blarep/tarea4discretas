\documentclass[spanish, fleqn]{article}
\usepackage{babel}
\usepackage[utf8]{inputenc}
\usepackage{amsmath,amsfonts}
\usepackage{enumitem}
\usepackage[colorlinks, urlcolor=blue]{hyperref}
\usepackage{fourier}
\usepackage{tikz}
\usetikzlibrary{shapes.geometric}
\usetikzlibrary{positioning}
\usepackage{verbatim}
\usepackage[top = 2.0cm, bottom = 2cm, left = 2.5cm, right = 2.5cm]{geometry}

\newcommand{\num}{4}

\title{Estructuras Discretas \\
	Tarea \#\num \\
	``Cómo aprender a generar en una tarde''}
\author{Andrés Navarro \\ (201673001-K)}
\date{}
\begin{document}
	\maketitle
	\thispagestyle{empty}
	
    Utilice funciones generatrices ordinarias para responder las siguientes preguntas.
    
	\section{Recurrencia}
	Halle una fórmula explícita para \(a_n\) dada la siguiente recurrencia:
    \begin{equation*}
    a_{n} = {n \choose 2} + 3a_{n-1} \qquad a_{0}=1
    \end{equation*}
    
    Se define la generatriz a buscar:
    \begin{equation*}
    A(z) = \sum_{n \geqslant 0} a_n z^n
    \end{equation*}
	
	Modificamos la expresión inicial:
	\begin{equation*}
	a_{n+1} = {n+1 \choose 2} + 3a_{n} \qquad a_{0}=1
	\end{equation*}
	
	Aplicando ogf en cada término:
	\begin{align*}
	\sum_{n \geqslant 0} a_{n+1}z^n &= \sum_{n \geqslant 0} {n+1 \choose 2}z^n +\sum_{n \geqslant 0} 3a_{n}z^n \\
	\frac{A(z)- a_{0}}{z} &T= \sum_{n \geqslant 0} {n+1 \choose 2}z^n + 3A(z)
	\end{align*}
	%% Pregunta 2
	\section{Safari}
	Un safari nuevo en la Selva Amazónica quiere armar un área mixta solo con animales herbívoros. El safari cuenta con los siguientes: elefantes, jirafas, ciervos y gansos. El área debe obedecer:
    \begin{itemize}
    \item Puede haber una cantidad libre de gansos.
    \item No puede haber más de dos elefantes.
    \item Los ciervos deben ir en grupos de a cuatro.
    \item No debe haber más de cinco jirafas.
    \end{itemize}
    \begin{enumerate}
    \item[a)] ¿De cuántas formas se puede armar esta área si se quiere que hayan 20 animales en ella?
    \item[b)] Se agrega una quinta regulación: si se decide incluir alguna especie en el área, se debe incluir más de un animal de dicha especie. ¿De cuántas formas se puede armar ahora el área con 20 animales?
    \end{enumerate}
	\hfill (35 ptos.)
	
	%% Pregunta 3
	\section{Da2 mágicos}
    Se tienen dos dados canónicos de 4 caras (\{1,2,3,4\}), un dado canónico de 6 caras (\{1,2,3,4,5,6\}), dos dados canónicos de 8 caras (\{1,2,3,4,5,6,7,8\}) y un dado desconocido. Determine, en cada caso, el dado desconocido (número de caras y los valores de cada cara):
    \begin{itemize}
    \item[a)] Si se tiran los 2 dados canónicos de 4 caras y el dado desconocido, existe la misma posibilidad de que salgan cada uno de los valores que al tirar los 2 dados canónicos de 8 caras.
    \item[b)] Determine si existe un dado tal que si se tira el dado desconocido y el dado (\{0,1,2\}), exista el doble de posibilidad de que salga un 1, 3 y 5 que al tirar el dado canónico de 6 caras, y la misma posibilidad para los demás valores.
    \item[c)] ¿Existe un dado tal que al tirarlo con el dado \{1,2,2,3,3,4,4,5\}, tiene la misma posibilidad de que salgan cada uno de los valores que al tirar los dos dados canónicos de 8 caras? Si es así, encuentre el dado.
	\end{itemize}
    
	\hfill (35 ptos.)\\
	\vfill\hfill DSW/\LaTeXe
\end{document}
